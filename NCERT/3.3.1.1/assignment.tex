\let\negmedspace\undefined
\let\negthickspace\undefined
\documentclass[journal]{IEEEtran}
\usepackage[a5paper, margin=10mm, onecolumn]{geometry}
%\usepackage{lmodern} % Ensure lmodern is loaded for pdflatex
\usepackage{tfrupee} % Include tfrupee package

\setlength{\headheight}{1cm} % Set the height of the header box
\setlength{\headsep}{0mm}     % Set the distance between the header box and the top of the text

\usepackage{gvv-book}
\usepackage{gvv}
\usepackage{cite}
\usepackage{amsmath,amssymb,amsfonts,amsthm}
\usepackage{algorithmic}
\usepackage{graphicx}
\usepackage{textcomp}
\usepackage{xcolor}
\usepackage{txfonts}
\usepackage{listings}
\usepackage{enumitem}
\usepackage{mathtools}
\usepackage{gensymb}
\usepackage{comment}
\usepackage[breaklinks=true]{hyperref}
\usepackage{tkz-euclide} 
\usepackage{listings}
% \usepackage{gvv}                                        
\def\inputGnumericTable{}                                 
\usepackage[latin1]{inputenc}                                
\usepackage{color}                                            
\usepackage{array}                                            
\usepackage{longtable}                                       
\usepackage{calc}                                             
\usepackage{multirow}                                         
\usepackage{hhline}                                           
\usepackage{ifthen}                                           
\usepackage{lscape}
\begin{document}

\bibliographystyle{IEEEtran}
\vspace{3cm}

\title{CHAPTER - 3\\Pair of Linear Equations in Two Variables}
\author{EE24BTECH11061 - Rohith Sai}
% \maketitle
% \newpage
% \bigskip
{\let\newpage\relax\maketitle}

\renewcommand{\thefigure}{\theenumi}
\renewcommand{\thetable}{\theenumi}
\setlength{\intextsep}{10pt} % Space between text and floats

\numberwithin{figure}{enumi}
\renewcommand{\thetable}{\theenumi}

\section*{Exercise : 3.3}
\begin{enumerate}
\item [1.1)] Solve the following pair of linear equations using LU decomposition:\\
\textbf{Solution:}\\
\begin{align}
    x + y &= 14 \\
    x - y &= 4
\end{align}

First, we rewrite the question as a system of linear equations.
\begin{align}
    x_1 &\implies x, \\
    x_2 &\implies y
\end{align}

Converting into matrix form, we get:
\begin{align}
    \myvec{1 & 1\\ 1 & -1}\myvec{x_1 \\ x_2} &= \myvec{14 \\ 4} \\
    \vec{A}x &= \vec{b}
\end{align}
To solve the above equation, we apply LU decomposition to matrix \(\vec{A}\).

We decompose \(\vec{A}\) as:
\begin{align}
    \vec{A} &= \vec{LU}, \\
    \vec{L} &= \text{Lower Triangular Matrix}, \\
    \vec{U} &= \text{Upper Triangular Matrix}.
\end{align}
Let \(y = \vec{U}x\), then we can rewrite the above equation as:
\begin{align}
    \vec{A}x = \vec{b} \implies \vec{LU}x = \vec{b} \implies \vec{L}y = \vec{b}
\end{align}
Now, the above equation can be solved using forward substitution since \(\vec{L}\) is lower triangular, thus we get the solution vector \(y\). Using this, we solve for \(x\) in \(y = \vec{U}x\) using back substitution knowing that \(\vec{U}\) is upper triangular.

LU Factorizing \(\vec{A}\), we get:
\begin{align}
    \vec{A} &= \myvec{1 & 0\\1 & -1}\myvec{1 & 1\\0 & -2}, \\
    \vec{L} &= \myvec{1 & 0\\1 & 1}, \\
    \vec{U} &= \myvec{1 & 1\\0 & -2}
\end{align}
The solution can now be obtained as:
\begin{align}
    \myvec{1 & 0\\1 & 1}\myvec{y_1 \\ y_2} &= \myvec{14 \\ 4}
\end{align}
Solving for \(y\), we get:
\begin{align}
    \myvec{y_1 \\ y_2} = \myvec{14 \\ -10}
\end{align}
Now, solving for \(x\) via back substitution:
\begin{align}
    \myvec{1 & 1\\0 & -2}\myvec{x_1 \\ x_2} &= \myvec{14 \\ -10}
\end{align}
\begin{align}
    x_2 &= 5, \\
    x_1 + x_2 &= 14 \implies x_1 = 9
\end{align}
Thus, the solution is:
\begin{align}
    x = 9, \; y = 5
\end{align}
\begin{figure}[H]
    \centering
    \includegraphics[width=\columnwidth]{figs/fig.png}
\end{figure}
\end{enumerate}
\end{document}
